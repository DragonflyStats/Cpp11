C++11 (formerly known as C++0x) is a version of the standard of the C++ programming language. It was approved by ISO on 12 August 2011, replacing C++03,[1] and superseded by C++14 on 18 August 2014.[2] The name follows the tradition of naming language versions by the year of the specification's publication.
 
C++11 includes several additions to the core language and extends the C++ standard library, incorporating most of the C++ Technical Report 1 (TR1) libraries — with the exception of the library of mathematical special functions.[3] C++11 was published as ISO/IEC 14882:2011[4] in September 2011 and is available for a fee. The working draft most similar to the published C++11 standard is N3337, dated 16 January 2012;[5] it has only editorial corrections from the C++11 standard.[6]
 
Work is currently under way on the C++17 standards.[7]

The modifications for C++ involve both the core language and the standard library.
 
In the development of every utility of the 2011 standard, the committee has applied some directives:
 Maintain stability and compatibility with C++98 and possibly with C
 Prefer introduction of new features through the standard library, rather than extending the core language
 Prefer changes that can evolve programming technique
 Improve C++ to facilitate systems and library design, rather than to introduce new features useful only to specific applications
 Increase type safety by providing safer alternatives to earlier unsafe techniques
 Increase performance and the ability to work directly with hardware
 Provide proper solutions for real-world problems
 Implement “zero-overhead” principle (additional support required by some utilities must be used only if the utility is used)
 Make C++ easy to teach and to learn without removing any utility needed by expert programmers
 
Attention to beginners is considered important, because they will always compose the majority of computer programmers, and because many beginners would not intend to extend their knowledge of C++, limiting themselves to operate in the aspects of the language in which they are specialized.[1]
