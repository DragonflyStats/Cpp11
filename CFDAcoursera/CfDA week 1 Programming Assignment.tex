\documentclass[a4paper,12pt]{article}
%%%%%%%%%%%%%%%%%%%%%%%%%%%%%%%%%%%%%%%%%%%%%%%%%%%%%%%%%%%%%%%%%%%%%%%%%%%%%%%%%%%%%%%%%%%%%%%%%%%%%%%%%%%%%%%%%%%%%%%%%%%%%%%%%%%%%%%%%%%%%%%%%%%%%%%%%%%%%%%%%%%%%%%%%%%%%%%%%%%%%%%%%%%%%%%%%%%%%%%%%%%%%%%%%%%%%%%%%%%%%%%%%%%%%%%%%%%%%%%%%%%%%%%%%%%%
\usepackage{eurosym}
\usepackage{vmargin}
\usepackage{amsmath}
\usepackage{graphics}
\usepackage{epsfig}
\usepackage{subfigure}
\usepackage{fancyhdr}
%\usepackage{listings}
\usepackage{framed}
\usepackage{graphicx}

\setcounter{MaxMatrixCols}{10}
%TCIDATA{OutputFilter=LATEX.DLL}
%TCIDATA{Version=5.00.0.2570}
%TCIDATA{<META NAME="SaveForMode" CONTENT="1">}
%TCIDATA{LastRevised=Wednesday, February 23, 2011 13:24:34}
%TCIDATA{<META NAME="GraphicsSave" CONTENT="32">}
%TCIDATA{Language=American English}

\pagestyle{fancy}
\setmarginsrb{20mm}{0mm}{20mm}{25mm}{12mm}{11mm}{0mm}{11mm}
\lhead{Dublin \texttt{R}} \rhead{Week 1}
\chead{Computing For Data Analysis }
%\input{tcilatex}


% http://www.norusis.com/pdf/SPC_v13.pdf
\begin{document}

% 1 - names - ready
% 2 - head - ready
% 3 - dim() -ready
% 4 - tail -ready
% 5 - Square Brackets -ready
% 6 - Missing Values - summary?
% 7 - summary (missing value remove) 
% 8 - subset + relational and logical operators
% 9 - str() - ready
%10 - subset according to category
 
\section{Reading in a CSV file}
% Section 1
To read in a csv file, it is convenient to save it to your \textit{\textbf{working directory}}, the default file directory that \texttt{R} uses. Once it is in the working directory, you can use the \texttt{read.csv()} command to load it into the \texttt{R} environment.
\begin{framed}
\begin{verbatim}
getwd()
#[1] "C:/Users/Computer5/Documents"

HW0 <- read.csv("hw0_data.csv",header = TRUE)
\end{verbatim}
\end{framed}
You can change the working directory using the \texttt{setwd()} to the one you require.
\begin{framed}
\begin{verbatim}
getwd()
#[1] "C:/Users/Computer5/Documents"

setwd("C:/Users/Computer5/Documents/R")

#  > getwd()
#  [1] "C:/Users/Computer5/Documents/R"
\end{verbatim}
\end{framed}




\section{Inspecting the data set}
% Section 2

\begin{itemize}
\item[2a)] Dimensions
\item[2b)] Column names (i.e. variables names)
\item[2c)] structure of data frame
\end{itemize}
%Programming Assignment Question 3
Lets compute the dimensions of the data frame \textit{\textbf{iris}}, and also the length of the \textit{\textbf{Nile}} data set.
\begin{framed}
\begin{verbatim}
dim(iris)
nrow(iris)
ncol(iris)
length(Nile)
\end{verbatim}
\end{framed}

Data frames often have specifically named rows and columns. Lets find out the names of the variables (columns) and cases (rows) for the data sets \textit{\textbf{iris}} and \textit{\textbf{mtcars}}.

%Programming Assignment Question 1
\begin{framed}
\begin{verbatim}
names(iris)
colnames(iris)
rownames(iris) # simply the case numbers.

names(mtcars)
rownames(mtcars)
colnames(mtcars)
\end{verbatim}
\end{framed}

\subsection{The \texttt{summary()} command}
The \texttt{summary()} command can be used to extract a short statistical summary (if applicable) from each column of the data frame. 
If there are missing values, the frequency of missing values will also be listed for each column.

\begin{verbatim}
> summary(iris)
  Sepal.Length    Sepal.Width     Petal.Length    Petal.Width          Species  
 Min.   :4.300   Min.   :2.000   Min.   :1.000   Min.   :0.100   setosa    :50  
 1st Qu.:5.100   1st Qu.:2.800   1st Qu.:1.600   1st Qu.:0.300   versicolor:50  
 Median :5.800   Median :3.000   Median :4.350   Median :1.300   virginica :50  
 Mean   :5.843   Mean   :3.057   Mean   :3.758   Mean   :1.199                  
 3rd Qu.:6.400   3rd Qu.:3.300   3rd Qu.:5.100   3rd Qu.:1.800                  
 Max.   :7.900   Max.   :4.400   Max.   :6.900   Max.   :2.500                
\end{verbatim}

\subsection{Types of Data in Dataframes}
%Programming assignment Question 9
What is the data type of each column in the iris data set? To find out, we use the \texttt{str()} command.

\begin{framed}
\begin{verbatim}
str(iris)
\end{verbatim}
\end{framed}
The output of this command is given below. There are four numeric variables, and one factor (i.e. categorical) variable.
\begin{verbatim}
'data.frame':   150 obs. of  5 variables:
 $ Sepal.Length: num  5.1 4.9 4.7 4.6 5 5.4 4.6 5 4.4 4.9 ...
 $ Sepal.Width : num  3.5 3 3.2 3.1 3.6 3.9 3.4 3.4 2.9 3.1 ...
 $ Petal.Length: num  1.4 1.4 1.3 1.5 1.4 1.7 1.4 1.5 1.4 1.5 ...
 $ Petal.Width : num  0.2 0.2 0.2 0.2 0.2 0.4 0.3 0.2 0.2 0.1 ...
 $ Species     : Factor w/ 3 levels "setosa","versicolor",..: 1 1 1 1 1 1 1 1 1 1 ...
> 
\end{verbatim}
%--------------------------------------%
\section{Reduction and Subsetting}
% Section 3
\begin{itemize}
\item[3a)] The \texttt{head()} and \texttt{tail()} function
\item[3b)] Accessing a particular row or set of rows
\item[3c)] subsetting with a relational condition.
\end{itemize}

%--------------------------------------%
% Programming Assignment Question 2 and 4
\subsection{The \texttt{head()} and \texttt{tail()} function}
The head and tail functions can be used to access the first six and last six rows of a data frame. If a different number of rows is required, all you have to do is specify that number as an additional argument.
\begin{framed}
\begin{verbatim}
head(iris)     #First 6 rows
head(iris,2)   #First 2 rows

tail(iris)     #Last 6 rows
tail(iris,4)   #Last 4 rows
\end{verbatim}
\end{framed}

%--------------------------------------%
% Programming Assignment Question 5
\subsection{Accessing a particular row or set of rows}

\begin{itemize}
\item Each value in a dataframe can be accessed directly by specifying the row and column i.e. \texttt{df[r,c]}.
\item To access a particular row, simply specify the row number, while leaving the column number blank i.e. \texttt{df[r,]}
\item To access a particular column, simple specify the column number, while leaving the row number blank i.e. \texttt{df[,c]}
\end{itemize}

\begin{framed}
\begin{verbatim}
iris[10,2]

iris[10,] 

Formaldehyde[,2]  
\end{verbatim}
\end{framed}

\begin{verbatim}
> iris[10,2]
[1] 3.1
> iris[10,] 
   Sepal.Length Sepal.Width Petal.Length Petal.Width Species
10          4.9         3.1          1.5         0.1  setosa
> 
> Formaldehyde[,2]  
[1] 0.086 0.269 0.446 0.538 0.626 0.782
\end{verbatim}
%--------------------------------------%
\section{Missing data}
%Section 4
\begin{itemize}
\item[(4a)] Determining the number of missing data items
\item[(4b)] Performing statistical operations removing missing data
\end{itemize}

As stated previously, the \texttt{summary()} command can be used to determine the number of missing data items in a data frame. The additional argument \texttt{na.rm=TRUE} can also be used with certain functions (see the help file)

\begin{verbatim}
> X <- c(4,6,3,12,NA,8)
> mean(X)
[1] NA
>
> summary(X)
   Min. 1st Qu.  Median    Mean 3rd Qu.    Max.    NA's 
    3.0     4.0     6.0     6.6     8.0    12.0       1
>
> mean(X ,na.rm = TRUE)
[1] 6.6
\end{verbatim}
\newpage
\section{Subsetting Data}
\subsection*{Logical and Relational Operator}

\begin{itemize}
\item AND - The logical operator is $\&$
\item OR - The logical operator is $\|$
\end{itemize}

\subsection*{Selection using the \texttt{subset()} Function}
The \texttt{subset( )} function is the easiest way to select variables and observation. 


\noindent \textbf{Example 1} \\
In the following example we will use the \textit{\textbf{iris}} data set, we select all rows that have a value of sepal length of 6 or more, and determine how many observations there are, and then compute the mean of the petal lengths. (The answer is 5.263, from the summary output).

\begin{framed}
\begin{verbatim}

#call the subset iris.2

iris.2 = subset(iris,iris$Sepal.Length >= 6)

dim(iris.2)

summary(iris.2)

\end{verbatim}
\end{framed}

\noindent \textbf{Example 2} \\
There are three types of iris - setosa, versicoloir and virginica. Suppose we wish to compute the median of petal widths for the setosa irises only.

The equality operator is $==$.
\begin{verbatim}
> iris.setosa =subset(iris,iris$Species=="setosa")
> summary(iris.setosa)
  Sepal.Length    Sepal.Width     Petal.Length    Petal.Width   
 Min.   :4.300   Min.   :2.300   Min.   :1.000   Min.   :0.100  
 1st Qu.:4.800   1st Qu.:3.200   1st Qu.:1.400   1st Qu.:0.200  
 Median :5.000   Median :3.400   Median :1.500   Median :0.200  
 Mean   :5.006   Mean   :3.428   Mean   :1.462   Mean   :0.246  
 3rd Qu.:5.200   3rd Qu.:3.675   3rd Qu.:1.575   3rd Qu.:0.300  
 Max.   :5.800   Max.   :4.400   Max.   :1.900   Max.   :0.600  
       Species  
 setosa    :50  
 versicolor: 0  
 virginica : 0 
\end{verbatim} 
 
\noindent \textbf{Example 3} \\
In the following example we will use the \textit{\textbf{iris}} data set, we select all rows that have a value of sepal length of 6 or more, but have sepal width is at least 2.5.

\begin{verbatim}
> iris.3 = subset(iris,(iris$Sepal.Length >= 6)&(iris$Sepal.Width >=2.5))
> summary(iris.3)
  Sepal.Length    Sepal.Width     Petal.Length    Petal.Width   
 Min.   :6.000   Min.   :2.500   Min.   :4.000   Min.   :1.200  
 1st Qu.:6.300   1st Qu.:2.800   1st Qu.:4.750   1st Qu.:1.500  
 Median :6.500   Median :3.000   Median :5.300   Median :1.800  
 Mean   :6.641   Mean   :3.013   Mean   :5.313   Mean   :1.848  
 3rd Qu.:6.900   3rd Qu.:3.200   3rd Qu.:5.750   3rd Qu.:2.150  
 Max.   :7.900   Max.   :3.800   Max.   :6.900   Max.   :2.500  
       Species  
 setosa    : 0  
 versicolor:21  
 virginica :42  
\end{verbatim}


%--------------------------------------%
\end{document}

\begin{verbatim}
> names(HW1)
[1] "Ozone"   "Solar.R" "Wind"    "Temp"    "Month"   "Day" 
\end{verbatim}

