
\documentclass[11pt]{article} % use larger type; default would be 10pt
\usepackage{framed}
\usepackage[utf8]{inputenc} % set input encoding (not needed with XeLaTeX)
\usepackage{geometry} % to change the page dimensions
\geometry{a4paper} % or letterpaper (US) or a5paper or....


\usepackage{graphicx} % support the \includegraphics command and options

% \usepackage[parfill]{parskip} % Activate to begin paragraphs with an empty line rather than an indent

%%% PACKAGES
\usepackage{booktabs} % for much better looking tables
\usepackage{array} % for better arrays (eg matrices) in maths
\usepackage{paralist} % very flexible & customisable lists (eg. enumerate/itemize, etc.)
\usepackage{verbatim} % adds environment for commenting out blocks of text & for better verbatim
\usepackage{subfig} % make it possible to include more than one captioned figure/table in a single float
% These packages are all incorporated in the memoir class to one degree or another...
\usepackage{framed}

%%% HEADERS & FOOTERS
\usepackage{fancyhdr} % This should be set AFTER setting up the page geometry
\pagestyle{fancy} % options: empty , plain , fancy
\renewcommand{\headrulewidth}{0pt} % customise the layout...
\lhead{}\chead{}\rhead{}
\lfoot{}\cfoot{\thepage}\rfoot{}

%%% SECTION TITLE APPEARANCE
\usepackage{sectsty}
\allsectionsfont{\sffamily\mdseries\upshape} % (See the fntguide.pdf for font help)
% (This matches ConTeXt defaults)

%%% ToC (table of contents) APPEARANCE
\usepackage[nottoc,notlof,notlot]{tocbibind} % Put the bibliography in the ToC
\usepackage[titles,subfigure]{tocloft} % Alter the style of the Table of Contents
\renewcommand{\cftsecfont}{\rmfamily\mdseries\upshape}
\renewcommand{\cftsecpagefont}{\rmfamily\mdseries\upshape} % No bold!

\begin{document}
\tableofcontents
\newpage
%------------------------------------------------------%
\section{Computing For Data Analysis Week 4}

\subsection{The \texttt{grep()} function }
 What does the \texttt{grep()} function do when called with its default arguments?
 
 It returns the indices for strings in a character that match a given regular expression.

\subsection{Regular Expressions}
%Question 7
\begin{framed}
\begin{verbatim}
^s(.*?)r
\end{verbatim}
\end{framed}
%------------------------------------------------------%
\subsection{Generic Functions}
%Question 8
In the R system of classes and methods, what is a generic function?


\subsection{S4 Methods}
What function is used to obtain the function body for an S4 method function?

Not \texttt{getS3method()}. 

What does the setOldClass function do?

\end{document}
