\subsection{Linear Algebra Functions}
R supports many import linear algebra functions such as cholesky decomposition, trace, rank, eigenvalues etc. The required results may be determinable from the output of a command that pertains to an overall approach.
The eigenvalues and eigenvectors can be computed using the eigen() function. 

A data object known as a list is then created.
eigen(A) #eigenvalues and eigenvectors qr(A) 
#returns Rank of a matrix svd(A)
This is a very important type of matrix analysis, and many will encounter it again in future modules. 

Y = eigen(A) 
names(Y) # y$val are the eigenvalues of A # y$vec are the eigenvectors of A



\subsection{Determinants, Inverse Matrices and solving Linear systems}
%================================================================================= %
To compute the determinant of a square matrix, we simply use the det() command
det(A)
det(B)
%================================================================================== %
To find the inverse of a square matrix, we use the solve() command, specifying only the matrix in question
solve(A)

To solve a system of linear equations in the form Ax=b , where A is a square matrix, and b is a column vector of known values, we use the solve() command to determine the values of the unknown vector x.
\begin{verbatim}
b=vec2  # from before
solve(A, b)
\end{verbatim}
%================================================================================== %
	Eigenvalues and Eigenvectors
	The eigenvalues and eigenvectors can be computed using the eigen() function.  A data object is created.
	This is a very important type of matrix analysis, and many will encounter it again in future modules.
	\begin{verbatim}
	Y = eigen(A)
	names(Y)
	"	y$val are the eigenvalues of A
	"	y$vec are the eigenvectors of A
	\end{verbatim}
		\subsection{Eigenvalues and Eigenvectors}
		The eigenvalues and eigenvectors can be computed using the eigen() function.  A data object is created.
		This is a very important type of matrix analysis, and many will encounter it again in future modules.
		\begin{verbatim}
		
		Y = eigen(A)
		names(Y)
		"	y$val are the eigenvalues of A
		"	y$vec are the eigenvectors of A
		
		?
		Part 2 Revision on Earlier Material
		"	Accessing a column of a data frame
		"	Accessing a row of a data frame
		\end{verbatim}